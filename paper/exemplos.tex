\begin{document}

% \begin{otherlanguage}{spanish}
% \begin{abstract}
%   Bienvenido a la clase \cls{rctart} \LaTeX\ para realizar trabajos académicos e informes de laboratorio. En esta plantilla de ejemplo, lo guiaremos a través del proceso de uso y personalización del documento según sus necesidades. Para obtener más información sobre esta clase, consulte la sección de apéndices. Allí encontrarás códigos que definen los aspectos principales del modelo, permitiéndote explorarlos y modificarlos. Si no necesita el resumen, comente el entorno \env{abstract}. Cabe mencionar que esta plantilla está inspirada en el trabajo de otros autores de la clase \LaTeX, la clase \cls{\href{https://memonotess1.wixsite.com/memonotess}{rho}} \LaTeX\, diseñado con intenciones académicas.
%    \printkeywords{palabra clave 1, palabra clave 2, palabra clave 3, palabra clave 4, palabra clave 5.} % debe poner \@keyword o \@wordskey o directamente el texto de las palabras clave
% \end{abstract}
% \end{otherlanguage}

%\begin{rctartenv}[frametitle=Atenção]
%Caixa azul de destaque 
%\end{rctartenv}

%\begin{figure}[htb]
%    \centering
%    \caption{Exemplo de figura obtido de PGFPlots \cite{PFGPlots}.\label{fig:colunasimples}}
%    \includegraphics[width=1.0\columnwidth]{figuras/exemplo.pdf}    
%    
%    \notetext{Fonte: Adaptado de \citeonline{PFGPlots}}
%\end{figure}
\appendix
\part*{Apêndices}

\section{TÍTULO}

O título do artigo (na primeira página) deve ser centralizado, com fontes maiusculas e tamanho Times New Roman, 14pt em negrito. Nome(s) dos autores e afiliações devem aparecer abaixo do título em letras maiusculas e minúsculas.


\section{Seções e Sub-seções}

 \subsection{Subseções}

Subseções devem aparecer em letras minúsculas (apenas a primeira letra da palavra inicial em maiuscula) em negrito e devem começar na margem esquerda em uma linha separada.

\subsubsection{Sub-subseções}

Sub-subseções, como no presente subitem, são desencorajadas. No entanto, se você precisar usá-las, estas devem aparecer em letras minúsculas (apenas a primeira letra da palavra inicial em maiuscula) e começar na margem esquerda em uma linha separada, com início do texto do parágrafo na linha seguinte. Os títulos das sub-subseções devem estar em itálico.

\section*{Seção Não Numerada} \label{sec:unsec}

    Se uma seção não numerada for declarada, um quadrado aparecerá seguido do nome da seção. Este estilo é característico desta classe e é apenas para seções de primeiro nível.

    Como isso afeta o título do índice e das referências, pode-se fazer uma modificação na seção \opt{style} \cls{rctart} para remover o quadrado. Consulte o apêndice para obter mais informações.


\section{Figuras e tabelas}

    \subsection{Exemplo de Figura}

        A Figura \ref{fig:figura3} mostra um exemplo de figura em uma coluna.
        
            \begin{figure}[htb]
                \centering
                \caption{Exemplo de figura obtido de PGFPlots \cite{PFGPlots}.}
                \includegraphics[width=0.71\columnwidth]{figuras/exemplo.pdf}
                \label{fig:figura3}
                
                
            \notetext{Fonte: Adaptado de ...}
            \end{figure}

    \subsection{Exemplo de Figura em Coluna Dupla}

    A Figura \ref{fig:exemplofloat} mostra um exemplo de uma figura flutuante de duas imagens que cobre a largura de duas colunas. Ele pode ser posicionado na parte superior ou inferior da página. O espaço entre as figuras também pode ser alterado utilizando o comando \lstinline[language=TeX]|\hspace{Xpt}|.

        \begin{figure*}[t!hb] % t para posição no topo da página atual; b para posição na parte inferior da página atual
            \centering
            \caption{Exemplo de figura que cobre a largura da página obtida em \citeonline{PFGPlots}.}
            \hfill
            \begin{subfigure}[b]{0.38\linewidth}% Fig (a)
                    \includegraphics[width=\linewidth]{figuras/exemplo2.pdf}
                    \caption{Exemplo de figura à esquerda.}
                    \label{fig:figa}
            \end{subfigure}
            \hfill % Espaço entre os algarismos
            \begin{subfigure}[b]{0.38\linewidth}% Fig (b)
                    \centering
                    \includegraphics[width=\linewidth]{figuras/exemplo2.pdf}
                    \caption{Exemplo de figura à direita.}
                    \label{fig:figb}
            \end{subfigure}
            \label{fig:exemplofloat}
            \hspace*{\fill}
            
            \notetext{Fonte: Adaptado de \citeonline{PFGPlots}.}
        \end{figure*}
        

   \subsection{Exemplo de Tabela}

        Da mesma forma que as figuras, pode-se colocar as tabelas em uma ou duas colunas, dependendo do comprimento da tabela.

        Tabela \ref{tab:tabela1}, mostra um exemplo de tabela que cobre a largura de uma colunas enquanto a tabela \ref{tab:tabela2} mostra um exemplo de tabela que cobre a largura de duas colunas
\newcolumntype{P}[1]{>{\centering\arraybackslash}p{#1}}
\newcolumntype{M}[1]{>{\centering\arraybackslash}m{#1}}
        \begin{table}[h]
            \RaggedRight\small
            \caption{Exemplo de tabela que cobre a largura da coluna.}
            \label{tab:tabela1}
            \centering
            \begin{tabular}{@{}lP{2.3cm}P{2.3cm}@{}}
                    \toprule
                    \textbf{Dia}    & \textbf{Temp. mínima}   & \textbf{Temp. máxima} \\ 
                    \midrule
                    Segunda-feira   & 11\textdegree C               & 22\textdegree C            \\
                    Terça-feira     & 9\textdegree C                & 19\textdegree C           \\
                    Quarta-feira    & 10\textdegree C               & 21\textdegree C           \\
                    \bottomrule
                \end{tabular}
                
            \notetext{Fonte: Adaptado de tabelas \LaTeX\ \cite{projects-2023}.}
            
        \end{table}
        
        \begin{table*}[h]
            \RaggedRight\small
            \caption{Exemplo de tabela que cobre a largura da página.}
            \label{tab:tabela2}
            \centering
            \begin{tabular}{lccp{7.5cm}}
                    \toprule
                    \multicolumn{1}{c}{\textbf{Dia}}    & \textbf{Temp. mínima}   & \textbf{Temp. máxima} & \multicolumn{1}{c}{\textbf{Resumo}} \\ 
                    \midrule
                    Segunda-feira   & 11 \textdegree C               & 22 \textdegree C           & Um dia claro com muito sol.  A brisa forte derrubará as temperaturas. \\
                    Terça-feira     & 9 \textdegree C                & 19 \textdegree C           & Nublado com chuva, em muitas regiões do norte. \\
                    Quarta-feira    & 10 \textdegree C               & 21 \textdegree C           & As chuvas ainda durarão pela manhã. 
                    As condições vão melhorar no início da tarde e continuar durante toda a noite.\\
                    \bottomrule
                    
                \end{tabular}
                
            \notetext{Fonte: Adaptado de tabelas \LaTeX\ \cite{projects-2023}.}
            
        \end{table*}
        
        
        
        Caso queira, use o site \href{https://www.tablesgenerator.com/}{https://www.tablesgenerator.com/}  para desenhar suas tabelas. Neste site, pode-se visualizar a tabela, alterá-la e copiar o código \LaTeX\ gerado e inserir aqui para compilação. 
        
        \section{Citações Diretas\label{sec:citacoes}}

 As citações diretas são feitas seguindo, mais ou menos, o que diz a norma NBR ABNT 10520 de \citeyear{NBR10520:2023} \cite{NBR10520:2023}. Para isso, esta classe disponibiliza o ambiente \env{citacao} como a seguir:
\begin{citacao}
 71.1 A citação direta, com mais de três linhas, deve ser destacada com recuo padronizado em relação à margem esquerda, com letra menor que a utilizada no texto, em espaço simples e sem aspas. Recomenda-se o recuo de 4 cm. \cite[p. 12]{NBR10520:2023}
\end{citacao}

O recuo deve ser de 4 cm se o documento estiver em coluna simples. Contudo, o recuo aqui será de 1,8 cm se estiver em coluna dupla, pois com 4 cm a citação não ficará confortável.

\section{Expressões matemáticas}

\subsection{Expressão na Linha}

Pode-se incluir expressões matemáticas na linha do texto como o Teorema de Pitágoras $c^2=a^2+b^2$ usando o código \lstinline[language=TeX]|$c^2=a^2+b^2$|  de maneira que a matemática esta integrada ao texto da linha de um parágrafo. 

É possíve colocar a Equaçao de Schrödinger em linha, mas não fica bonito. Veja que $\frac{-\hbar^2}{2m}\nabla^2\Psi + V\left(\vec{r}\right)\Psi = -i\hbar \frac{\partial\Psi}{\partial t}$ fica expremido.

\subsection{Expressão Ccentralizada Não Numerada}
Expressões matemáticas em linha separada e centralizadas mas sem numeração podem ser criadas como o Teorema de Pitágoras anterior, mas com o código \lstinline[language=TeX]|\[c^2=a^2+b^2\]| cujo resultado é \[c^2=a^2+b^2.\]

Assim como a Equaçao de Schrödinger \[-\frac{\hbar^2}{2m}\nabla^2\Psi + V\left(\vec{r}\right)\Psi = -i\hbar \frac{\partial\Psi}{\partial t}.\]

\subsection{Expressão Centralizada Numerada}
    O ambiente \env{equation} é para uma única equação com uma geração automática de numeração. O ambiente da \env{equation*} faz o mesmo, com a diferença de que ele omite a numeração dela.

    A equação \eqref{eq:equacao1}, mostra a equação de Schrödinger como exemplo utilizando o ambiente \env{equation}. 
        \begin{equation} \label{eq:equacao1}
            -\frac{\hbar^2}{2m}\nabla^2\Psi + V\left(\vec{r}\right)\Psi = -i\hbar \frac{\partial\Psi}{\partial t}
        \end{equation} 
        
\subsection{Expressão Centralizada Multi-linha}

        Equações multilinha podem ser feitas com os ambientes \env{eqnarray} ou \env{align} e com o \env{multline}. 
        
        A equação \eqref{eq:equacao2} mostra o ambiente matemático  \env{eqnarray} para a passagem da equação de Schrödinger dependente do tempo para a indenpendente do tempo. Considerando a separação de variáveis 
        $$\Psi(\vec r,t)=\psi(\vec r)e^{-iEt/\hbar}$$
        a derivada temporal 
        \begin{eqnarray}\label{eq:equacao2}
         -i\hbar\frac{\partial\left[\psi\left(\vec{r}\right)e^{-iEt/\hbar}\right]}{\partial t}&=&-i\hbar\psi\left(\vec{r}\right)\frac{\partial e^{-iEt/\hbar}}{\partial t}\nonumber\\
         &=&E\psi\left(\vec{r}\right)e^{-iEt/\hbar}
        \end{eqnarray}
        
        Para comparação, a mesma expressão é feita na equação \eqref{eq:equacaoalign} com o ambiente \env{align} que a deixa mais compacta e tipograficamente confortável
        \begin{align}\label{eq:equacaoalign}
         -i\hbar\frac{\partial\left[\psi\left(\vec{r}\right)e^{-iEt/\hbar}\right]}{\partial t}&=-i\hbar\psi\left(\vec{r}\right)\frac{\partial e^{-iEt/\hbar}}{\partial t}\nonumber\\
         &=E\psi\left(\vec{r}\right)e^{-iEt/\hbar}.
        \end{align}

        
        A  equação \eqref{eq:equacao3} mostra o ambiente \env{multline} para 
         \begin{multline} \label{eq:equacao3}
            -\frac{\hbar^2}{2m}\nabla^2\left[\psi(\vec r)e^{-iEt/\hbar}\right] + V(\mathbf{r})\psi(\vec r)e^{-iEt/\hbar} \\
            =\left[\frac{-\hbar^2}{2m}\nabla^2\psi(\vec r) + V(\mathbf{r})\psi(\vec r)\right]e^{-iEt/\hbar} \\
            = -i\hbar \frac{\partial\left[\psi(\vec r)e^{-iEt/\hbar}\right]}{\partial t}\\
            = -i\hbar \psi(\vec r)\frac{\partial e^{-iEt/\hbar}}{\partial t}\\
            = E \psi(\vec r) e^{-iEt/\hbar}.
        \end{multline}
%         ]
        Resultando na equação de autovalor \eqref{eq:equacao4}
        \begin{equation}\label{eq:equacao4}
         -\frac{\hbar^{2}}{2m}\nabla^{2}\psi\left(\vec{r}\right)+V\left(\vec{r}\right)\psi(\vec{r})=E\psi\left(\vec{r}\right).
        \end{equation}

        Caso queira alterar os valores que ajustam o espaçamento acima e abaixo nas equações, vá até a seção de matemática de \filn{rctart-class/rctart.cls/} e brinque com o valor  \lstinline[language=TeX]|\setlength{\eqskip}{8pt}|  até que o espaçamento preferido seja definido.
        
        Há outros ambientes para tipografia de expressões matemáticas. Caso tenho curiosidade, veja os manuais do pacote \nipkg{amsmath}.

\section{Códigos}

    Esta classe \footnote{Olá! Eu sou uma nota de rodapé :)} inclui o pacote \nipkg{listings}, que oferece recursos customizados para adicionar códigos especialmente para C, C++, \LaTeX\ e Matlab. Pode-se personalizar o formato no arquivo da classe \cls{rctart}. O Código \ref{lst:listing-Mat} mostra com fica a tipografia.

    \nolinenumbers
    \lstinputlisting[language=Matlab,caption=Exemplo de código matlab., label={lst:listing-Mat}]{example.m}
    \linenumbers
    
    O Código \ref{lst:listing-latex1} é um exemplo de como se escreve um código com legenda (caption) e rótulo (label), e isere o conteúdo do código que está em um arquivo externo \filn{example.m}.
    \nolinenumbers
\begin{lstlisting}[language=TeX,caption=Exemplo de listings com código em arquivo externo,label={lst:listing-latex1}]
\lstinputlisting[language=Matlab,caption=Exemplo de código matlab.,label={lst:listing-Mat}]{example.m}
\end{lstlisting}
    
     O Código \ref{lst:listing-latex2} é um exemplo de como se escreve um código com legenda (caption) e rótulo (label), e isere o conteúdo diretamente no ambiente.
    \nolinenumbers
\begin{lstlisting}[language=TeX,caption=Exemplo de listings com código em arquivo externo,label={lst:listing-latex2}]
\begin{lstlisting}[language=TeX,
    caption=Exemplo de listings 
             de arquivo externo,
    label={lst:latex1}]
         O código entra aqui.
\end{lstlisting }
\end{lstlisting}
    
    
    Se a numeração de linhas estiver habilitada, recomendamos colocar o comando \lstinline[language=TeX]|\nolinenumbers| no início e \lstinline[language=TeX]|\linenumbers| no final do código. 
    
    Isso removerá temporariamente a numeração das linhas e o código ficará melhor.
    
    Foram introduzidos alguns comandos para a tipografia de nomes de arquivos de classes, pacotes, ambientes, fontes de letras, opções e arquivos que estão por padrão em 'typewriter' como no Código \ref{lst:listing-latex3}.
\begin{lstlisting}[language=TeX,caption={Tipografia de nomes de arquivos, ambientes, fontes de letras, etc},label={lst:listing-latex3}]
\newcommand{\cls}{\texttt}
\newcommand{\nipkg}{\texttt}
\newcommand{\env}{\texttt}
\newcommand{\fnt}{\texttt}
\newcommand{\opt}{\texttt}
\newcommand{\filn}{\texttt}
\end{lstlisting}



\section{Sumário/ToC}

    O ToC fornece uma visualização do conteúdo e sua localização no documento. Remova o comentário do comando \lstinline[language=TeX]|\tableofcontents| para exibi-lo. Lembre-se que seções não numeradas não aparecerão no ToC, porém, pode-se colocá-las manualmente com o comando 
\begin{lstlisting}
\addcontentsline{toc}{seção}{nome da seção}|. 
\end{lstlisting}


    Consulte a seção de apêndice para obter mais informações. Lá encontram-se as modificações recomendadas para ajustar o índice quando seções não numeradas forem definidas.

\section{Estilo de Referência}

    A formatação padrão para citação é o Autor-ano e as referências seguem o estilo ABNT NBR 10520, Citações em documentos \cite{NBR10520:2023} e ABNT NBR 6023, Referências \cite{NBR6023:2018}. 
    
    Pode-se modificar o estilo de suas referências, para isso acesse rct-class/rct.cls/rct.cls. Consulte o apêndice para obter mais informações.
	


    \section{Outros Elementos}

    \subsection{Lettrine}
    
        O comando \lstinline[language=TeX]|\rctartstart{}| fornece uma letra personalizada para o início de um parágrafo, conforme mostrado neste exemplo de documento.

    \subsection{Numeração de Linha}

        Ao implementar o pacote \nipkg{lineno}, a numeração das linhas do documento pode ser colocada usando-se as declarações \lstinline[language=TeX]|\linenumbers| e \lstinline[language=TeX]|\nolinenumbers| nos trechos que se deseja numerar as linhas. Esse recurso é útil para os avaliadores e corretores indicarem em que linha do documento PDF se encontra algum equívoco, imprecisão ou sugestão que deva ser inserida.

        Por padrão estão habilitados com \opt{linenumbers=on}, no entanto, pode-se desabilitar a numeração definindo \lstinline[language=TeX]|\documentclass[..,linenumbers=off,...]|.

        A numeração desaparecerá quando o documento for recompilado e nenhuma modificação no documento da classe será necessária.
        
        \subsection{Cores}
        
        A classe \cls{rctart} tipografa o texto em preto exceto o título do artigo, os título de seção primária (seções), os nomes de figuras, nomes de tabelas, nomes de códigos, hiperlinks, texto dentro de caixas de nota e de informação que são tipografados na cor base \opt{rctartcolor}.
        
        A cor base, \opt{rctartcolor}, que é usada nos itens descritos aneriomente, está definida por padrão como sendo \textit{Dark Midnight Blue}\footnote{Para mais cores, veja \href{https://latexcolor.com/}{https://latexcolor.com/}}, que no modelo rgb\footnote{No modelo rgb, cada cor pode assumir um valor no intervalo $[0,1]$, no modelo RGB, cada cor pode assumir um valor do conjunto $\{0,1,\ldots,L\}$, o padão é $L=255$ } é \opt{rgb=\{0.0, 0.2, 0.4\}}. 
        
        Esta cor base pode ser alterada inserindo, por exemplo, o Código \ref{codigocores} no preâmbulo, que altera para Verde Musgo identificada em RGB.:
\begin{lstlisting}[language=TeX, caption=Alteração da cor base,label=codigocores]
\definecolor{rctartcolor}
  {rgb}{0.12, 0.3, 0.17}  %Green
\end{lstlisting}

Fique a vontade para experimentar cores diferentes, mas o artigo para a Semana da Facet deve ser o da classe \cls{rctart}.

 \subsection{Seções Não Numeradas}

        Como mencionado na seção \ref{sec:unsec}, ao colocar uma seção de primeiro nível sem número aparece um quadrado seguido do nome da seção. Caso não necessite deste detalhe extra, pode-se fazer a seguinte modificação.

\nolinenumbers
\begin{lstlisting}[language=TeX, caption=Seção alternativa não numerada.]
\titleformat{nome=\section,numberless}[block]
    {\color{rctartcolor}\sffamily\large\bfseries}
    {}
    {0em}
    {#1}
    []
\end{lstlisting}
\notetext{Fonte: Adaptado de ...}
\linenumbers

        Pode-se alterar este código na seção \opt{section} em \filn{rctart-class/rctart.cls}. Assim que o documento for recompilado, este quadrado desaparecerá. 

        Lembre-se de que este código afeta o ToC e o título das referências. Para mostrar as funcionalidades da classe \cls{rctart}, esta opção está habilitada por padrão.

    \subsection{Sumário}
    
        Caso  tenha-se escolhido as seções não numeradas e queira-se adicionar o Sumário,  pode-se fazer o seguinte para ajustar o conteúdo.

\nolinenumbers
\begin{lstlisting}[language=TeX, caption=ToC quando seção não numerada é escolhida.]
\setlength\tocsep{0pc}

\titlecontents{section}[\tocsep]
    {\addvspace{4pt}\sffamily\selectfont\bfseries}
    {\contentslabel[\thecontentslabel]{\tocsep}}
    {}
    {\hfill\thecontentspage}
    []

\titlecontents{subsection}[1pc]
    {\addvspace{4pt}\small\sffamily\selectfont}
    {\contentslabel[\thecontentslabel]{\tocsep}}
    {}
    {\ \titlerule*[.5pc]{.}\ \thecontentspage}
    []

\titlecontents*{subsubsection}[1pc]
    {\footnotesize\sffamily\selectfont}
    {}
    {}
    {}
    [\ \textbullet\ ]
\end{lstlisting}
\linenumbers

        Como pode-se observar, o valor de \lstinline[language=TeX]|\tocsep| foi alterado para 0pc para as seções. Para subseções e subseções o valor foi alterado para 1pc.

        Ao fazer esta pequena modificação, o conteúdo do ToC ficará mais organizado.

        Usam-se seções numeradas, não será necessário fazer essas modificações, a menos que prefira-se outros valores.

    \subsection{Referências e Caminhos}

        Caso precise de outro estilo de referência, pode-se acessar a seção \opt{biblatex} em \filn{rctart-class/rctart.cls} e modificar o seguinte.

\nolinenumbers
\begin{lstlisting}[language=TeX, caption=Estilo de referência.]
\RequirePackage[
% 	backend=biber,
	style=abntex2-alf,
% 	sorting=ynt
]{biblatex}
\end{lstlisting}
\linenumbers

        Por padrão, \cls{rct class} tem seu próprio .bib para este exemplo, se for desejado nomear seu próprio arquivo \filn{bib}, altere o \opt{bibresource} se usar Bib\LaTeX:		
\nolinenumbers
\begin{lstlisting}[language=TeX]
\addbibresource{rct.bib}
\end{lstlisting}
\linenumbers
ou se usar o Bib\TeX:
\nolinenumbers
\begin{lstlisting}[language=TeX]
\bibliography{rct.bib}
\end{lstlisting}
\linenumbers

  \subsection{Ambiente de Informação e Nota}

        Mostraremos um exemplo do ambiente \env{info} declarado no pacote \nipkg{rctartenvs}. Lembre-se que \env{info} e \env{note} são os únicos pacotes que traduzem seus títulos (português ou espanhol).

        \begin{info}
            Pequeno exemplo de ambiente de informação.
        \end{info}
        \begin{note}
            Pequeno exemplo de ambiente de nota.
        \end{note}
        
        
        

    \section{Pacotes Rctart}

    \subsection{Rctartenvs}
    
        Este modelo possui seu próprio pacote \nipkg{rctartenvs} de ambientes, projetado para aprimorar a apresentação de informações em documentos. Entre esses ambientes customizados estão \env{rctartenv}, \env{info} e \env{note}.
    
        Existem dois ambientes que possuem um título predefinido. Estes podem ser incluídos pelos ambientes \env{note} e \env{info}, sendo iniciados com \lstinline[language=TeX]|\begin{note}| e \lstinline[language=TeX]|\begin{info}|, respectivamente. Todos os ambientes têm o mesmo estilo tipográfico.
    	
    	Um exemplo usando o ambiente \env{rctartenv} é mostrado abaixo.
    
        \begin{rctartenv}[frametitle=Ambiente com título personalizado]
            Olá! Eu sou um exemplo do \env{rctartenv} incluído no pacote \nipkg{rctartenvs}. Aqui  pode-se incluir informações ou notas relevantes sobre o seu trabalho. Pode-se modificar meu título diretamente no código.
        \end{rctartenv}
    
        \env{Rctartenv} é o único ambiente cujo título pode-se personalizar. Por outro lado, \env{info} e \env{note} adaptam seu título para português automaticamente quando o pacote de idiomas \nipkg{babel} é definido.

    \subsection{Rctartbabel}

        Nesta nova versão, incluímos um pacote  \nipkg{rctartbabel}, que contém todos os comandos que traduzem automaticamente do inglês para o português quando o pacote de idiomas é definido.
\begin{note}
         Por padrão, \cls{rctart} exibe seu conteúdo em português, porém, pode-se definir o idioma no preâmbulo com \lstinline[language=TeX]|\selectlanguage{idioma}|, mas deve-se incluir \opt{idioma} no \lstinline[language=TeX]|\documentclass[...,idioma,...]|
\end{note}

        Pode-se modificar o pacote \nipkg{rctartbabel} se for necessário outro idioma. Isto tornará mais fácil traduzir o documento sem ter que modificar a classe de documento.
    
    
%----------------------------------------------------------
        
        
\section*{Entre em contato conosco}

    Divirta-se escrevendo com a classe  \LaTeX\  \cls{rctart} \hspace{5pt}\faHandSpock[regular] \\%\faChessKnight \\ \faLinux
%     \noindent\faWix\hspace{5pt}\href{https://memonotess1.wixsite.com/memonotess}{https://memonotess1.wixsite.com/memonotess} \\
    \faEnvelope[regular]\hspace{7pt}\href{mailto://silvio.granja@unemat.br}{silvio.granja@unemat.br}\\
    \faUniversity\hspace{7pt}\href{http://unemat.br}{http://unemat.br}\\
    \faMapMarked\hspace{7pt}\href{https://maps.app.goo.gl/d1oovfssBBWaKnRk6}{http://unemat.br}
%     \faEnvelope[normal]\hspace{7pt}eduardo.gracidas29@gmail.com \\
%     \faInstagram\hspace{8pt}memo.notess

\end{document}
